\documentclass[12pt]{article}

\begin{document}

\begin{flushright}
    \textsc{\LARGE Practical Anarchism}\\
    Winter 2023, Jichao Yang
\end{flushright}

\section{Introduction to Anarchism}

We will be spending most time with Branson's book \textit{Practical Anarchism},
and other readings will touch on history of anarchism, anarchism in feminism, environmental justice, and racial struggles. The accent of this course will be Anarchism as a \textbf{complex living force}, but we will not of course be neglecting the history of anarchism.\\

Other books and resources listed outside of the required readings might be useful for affinity group researches:
\begin{enumerate}
    \item \textit{No Gods, No Masters}, discusses the history and anthology of revolutions and political Anarchism.
    \item \textit{Pierre-Joseph Proudhon}, who famously said 'property is theft,' will be lightly discussed. But his anti-semitism and advocacy for patriarchy makes him not so fit for discussions of anarcho-feminism.
    \item \textit{Mikhail Bakunin} raises vehement objections against orthodox marxism. See more of his work in \textit{Roads to Freedom, Betrand Russell}.
    \item \textit{Peter Kropotkin} similarly attacks the Marxist theory of \textit{dictatorship of the proletariate} and argues a social movements should aim at the end result: his rejection of the orthodox Marxist version of dialectics entails a rejection of Marx's vision of a classless society. He also was interested in late Darwin focusing on natural mutual aid, and later studied geology and nature in eastern siberia and manchuria.
    \item \textit{Emma Goldman} brought the discussion of gender and sexual identity into the discussion of anarchism. More information on her can be found on PBS:\\
    \textit{www.pbs.org/wgbh/americanexperience/features/goldman-1869-1940}
\end{enumerate}

\section{Week 2 Title}

\end{document}