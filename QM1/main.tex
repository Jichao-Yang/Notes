\documentclass[12pt]{article}

\usepackage[a4paper, total={6in, 8in}]{geometry}
\usepackage{amsfonts}
\usepackage{amssymb}
\usepackage{latexsym}
\usepackage{amsthm}
\usepackage{amsmath}
\usepackage{graphicx}
\usepackage{tikz}
\usepackage{pstricks}
\usepackage{mathrsfs}


\iffalse

\def\T{{\mathbb{T}}}
\def\E{{\mathbb{E}}}
\def\Q{{\mathbb{Q}}}
\def\Qp{{\mathbb{Q}_p}}
\def\Qx{{\mathbb{Q}^\times}}
\def\Qpx{{\mathbb{Q}^\times_p}}
\def\Z{{\mathbb{Z}}}
\def\N{{\mathbb{N}}}
\def\A{{\mathbb{A}}}
\newcommand{\ds}{\displaystyle}
\newcommand{\vx}{\mathbf x}
\newcommand{\vy}{\mathbf y}
\newcommand{\vz}{\mathbf z}
\newcommand{\va}{\mathbf a}
\newcommand{\vb}{\mathbf b}
\newcommand{\vc}{\mathbf c}
\newcommand{\inn}[1]{\langle #1 \rangle}



\newtheorem*{theorem*}{Theorem}
\newtheorem{proposition}[theorem]{Proposition}
\newtheorem{lemma}[theorem]{Lemma}
\newtheorem{definitions}[theorem]{Definitions}
\newtheorem*{definition*}{Definition}
\newtheorem{exercise}[theorem]{Exercise}
\newtheorem{exercises}[theorem]{Exercise}



\fi

\def\R{{\mathbb{R}}}
\def\C{{\mathbb{C}}}
\newcommand{\norm}[1]{|| #1 ||}

\newtheorem{theorem}{Theorem}
\numberwithin{theorem}{section}
\newtheorem{remark}[theorem]{Remark}
\theoremstyle{definition}
\newtheorem{definition}[theorem]{Definition}
\newtheorem{example}[theorem]{Example}
\newtheorem{corollary}[theorem]{Corollary}



\begin{document}

\begin{flushright}
    \textsc{\LARGE Quantum Mechanics I}\\
    Winter 2023, Jichao Yang
\end{flushright}

\section{Vectors and Dirac Notation}

\begin{remark}
    This course will use two books, Shanker is more advanced and will be assigned with reading, Griffiths has more problem and easy-going in terms of difficulty.
\end{remark}

\begin{definition}
    A vector space V on a field $\mathbb{F}$ is a set of vectors that are closed under addition and multiplication. That is, for any $v, w\in V$, $a,b\in\mathbb{F}$, $av+bw\in V$.
\end{definition}

\begin{remark}
    From now on we will use $V$ to denote a vector space, $a,b,c,\dots$ to denote elements of $\mathbb{F}$, and $u,v,w,\dots$ to denote elements of $V$.
\end{remark}

\begin{definition}
    Vector addition and multiplication follow a set of axioms, which can be found in Shanker 1.1. Some of the most important ones are listed below:
    \begin{enumerate}
        \item $v+w = w+v$
        \item $(u+v)+w = u+(v+w)$
        \item $a\cdot(v+w) = a\cdot v + a\cdot w$
    \end{enumerate}
\end{definition}

\begin{definition}
    A real number space is a $V$ over $\mathbb{R}$. A complex number space is a $V$ over $\mathbb{C}$.
\end{definition}

\begin{remark}
    From now on we will be limiting our discussion to complex number spaces.
\end{remark}

\begin{example}
    Complex column vectors:    
    \begin{align}
        \begin{pmatrix}
            x_1 \\
            \vdots \\
            x_k
        \end{pmatrix}
    \end{align}
    Each $x_i\in \C$. Check that both operations are closed.
\end{example}

\begin{example}
    Complex function vectors: $F = \{f:[0,L]\subset\R\to\C\}$. Define addition and scalar multiplication as function addition and multiplying function with scalar. Check function is closed under both operations.
\end{example}

\begin{definition}
    $v_1, \dots, v_n$ are linearly dependent if exists $a_1, \dots, a_n$, at least one of them non-zero, such that
    $$\sum_{i=1}^{n} a_i\cdot v_i = 0$$\\
    $v_1, \dots, v_n$ are linearly independent otherwise.
\end{definition}

\begin{definition}
    The dimension of a vector field $V$ is the maximum number of linearly independent vectors subset to $V$.
\end{definition}

\begin{remark}
    Example 1.7 has dimension $k$, example 1.8 has infinite dimension.
\end{remark}

\begin{definition}
    $v_1, \dots, v_n$ span $V$ if and only if exists $a_1, \dots, a_n$, at least one nonzero, such that
    $$a_1v_1 + \dots + a_nv_n = 0$$
\end{definition}

\begin{definition}
    $v_1, \dots, v_n$ is a basis of $V$ if and only if they span $V$ and are linearly independent.
\end{definition}

\begin{corollary}
    If $v_1,\dots,v_n$ are a basis for $V$, then any $v\in V$ can be uniquely expressed by a linear combination of $v_1,\dots,v_n$.
\end{corollary}

\begin{definition}
    The complex conjugate of some complex number $a+bi$ is $a-bi$.    
\end{definition}

\begin{definition}
    An inner product function is $(\cdot):V\times V\to\C$, denoted as $(w,v)$ such that
    \begin{enumerate}
        \item $(w, a_1 v_1 + a_2 v_2) = a_1 \cdot (w, v_1) + a_2 \cdot (w, v_2)$
        \item $(v,w) \geq 0$, and only equal if $v$ or $w$ equals 0.
        \item $(w,v) = (v,w)^*$
    \end{enumerate}
\end{definition}

\begin{corollary}
    $(a_1 w_1 + a_2 w_2, v) = a_1^* \cdot (w_1,v) + a_2^* \cdot (w_2, v)$
\end{corollary}

\begin{proof}
    First note that $(c_1\cdot c_2)^* = (c_1)^* \cdot (c_2)^*$.
    \begin{align*}
        ((a+bi)(c+di))^* &= (ac-bd + (ad+bc)i)* \\
        &= ac-bd - (ad+bc)i = (a-bi)(c-di)\\
        &= (a+bi)^* (c+di)^*
    \end{align*}
    Hence the proof:
    \begin{align*}
        (a_1 w_1 + a_2 w_2, v) &= (v, a_1 w_1 + a_2 w_2)^*\\
        &= (a_1 (v, w_1) + a_2 (v, w_2))^* = (a_1 (v, w_1))^* + (a_2 (v, w_2))^*\\
        &= a_1^* (v, w_1)^* + a_2* (v, w_2)^* = a_1^* (w_1, v) + a_2^* (w_2, v)
    \end{align*}
\end{proof}

\begin{example}
    The inner product of example 1.7 is $(w,v) = w_1^* v_1 + \dots + w_k^* v_k$.
\end{example}

\begin{example}
    The inner product of example 1.8 is
    $$(f,g) = \int_{0}^{L} f^*(x)\cdot g(x) dx$$
\end{example}

\begin{definition}
    The norm (sometimes called length) of $v$ denoted $\norm{v}$ is $$\norm{v} = \sqrt{(v,v)}$$
\end{definition}

\begin{remark}
    The inner product is a function that maps to $\C$, but $(v,v)\in\R$, so the square root for Definition 1.20 is the real number square root.
\end{remark}

\begin{definition}
    $v,w$ are orthogonal if and only if $(v,w) = 0$.
\end{definition}

\begin{definition}
    $v_1, \dots, v_n$ are orthonormal if and only if for any $v_i, v_j$,
    $$(v_i, v_j) = \delta_{ij} = bool(m==n)$$
    Where $\delta_{ij}$ is called the 'kronecker delta.'
\end{definition}

\begin{corollary}
    $v_1, \dots, v_n$ are orthonoral if and only if they are orthognal and have length 1.
\end{corollary}

\begin{remark}
    From now on we will limit the discussion to vector spaces with well-defined inner-products.
\end{remark}

\begin{definition}
    A linear functional $F$ is a linear function $F:V\to\C$.
\end{definition}

\begin{corollary}
    The set of linear functionals $\{F\}$ form a vector space.
\end{corollary}

\begin{proof}
    Check closedness over addition and multiplication of linear functions.
\end{proof}

\begin{definition}
    Given any vector space $V$, exists a corresponding vector space $V'$, named the dual space, which is the set of all linear functions on $V$.
\end{definition}

\end{document}